\chapter*{Conclusiones y Recomendaciones}
\label{conclusiones}
\lhead{\emph{Conclusiones y Recomendaciones}}
\addcontentsline{toc}{chapter}{Conclusiones y Recomendaciones}

En este trabajo se plantea un estudio no antes realizado sobre la afectación del comportamiento de varias metaheurísticas al filtrar su población inicial. El filtro se realizó a través de heurísticas con comportamientos y tiempos particulares. El objetivo del trabajo fue determinar si las heurísticas benefician a las metahuerísticas. Para lograr dicho objetivo, se realizó un diseño de experimento donde se comparan todas las combinaciones a través de las métricas de \emph{accuracy} y reducción.


De las metahuerísticas seleccionadas, combinadas con las heurísticas escogidas, se puede concluir que la utilización de estas heurísticas no conllevan a una mejoría significativa. En el caso de GGA, el uso de heurísticas tiende a desfavorecer una métrica sobre otra, esto se aprecia para ENN, que obtiene los niveles más altos de \emph{accuracy y kappa} pero los niveles más bajos de reducción y viceversa para CNN. En el caso de MA y CHC se observan resultados similares entre todas las combinaciones, lo que reafirma el hecho de que ambas metaheurísticas consiguen soluciones parecidas de manera consistente, independientemente de la población inicial. Sólo SSGA se benefició del uso de heurísticas, ya que obtuvo mejores tasas de reducción con RSS, aumentando en promedio 5.65\% sobre lo obtenido por SSGA con población inicial aleatoria.

Otro punto que se deriva, es que los resultados obtenidos por el uso individual de las heurísticas (ver sección 3.2.1) no se correlacionan directamente con los resultados obtenidos por el uso combinado con las metahuerísticas (ver sección 3.2.3), ya que, entre las heurísticas, CNN resultó ser la mejor opción para los conjuntos de los tres tamaños, presentando los tiempos más cortos y sin embargo, al combinarse con las metaheurísticas, obtuvo tiempos más largos que los obtenidos por ENN y RSS en la mayoría de los casos, además de perder \emph{accuracy y kappa} en el caso de GGA. Se concluye entonces, que el resultado individual de una heurística no es un indicativo certero de su comportamiento al combinarse con las metaheurísticas. 

Finalmente, cuando se comparan todas las metaheurísticas, se tiene que las variantes de CHC y MA presentan los mejores valores de \emph{accuracy, kappa} y reducción para los conjuntos de los tres tamaños, además, estas metaheurísticas son estadísticamente similares. Por lo tanto, la elección de cuál es la mejor metahuerística para resolver el problema de selección de prototipos se rige por el tiempo de cómputo que toman en terminar. Ya que CHC con población inicial aleatoria es la metaheurística que tarda menos en devolver un resultado, se considera como la mejor. 

Las recomendaciones que se desprenden de esta investigación incluyen: 

\begin{itemize}

\item Probar otras heurísticas y metaheurísticas a las utilizadas en este trabajo.

\item Utilizar otras funciones de distancia distintas a la euclídea con el fin de incorporar conjuntos de datos categóricos.

\item Plantear el problema de selección de prototipos como un problema de optimización multiobjetivo en relación al \emph{accuracy, kappa}, reducción y tiempo para estudiar el comportamiento de los distintos algoritmos frente a la curva pareto obtenida.

\end{itemize}

