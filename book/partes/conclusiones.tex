\chapter*{Conclusiones y Recomendaciones}
\label{conclusiones}
\lhead{\emph{Conclusiones y Recomendaciones}}
\addcontentsline{toc}{chapter}{Conclusiones y Recomendaciones}

En este trabajo se plantea un estudio no antes realizado sobre la afectación del comportamiento de varias metaheurísticas al filtrar su población inicial. El filtro se realizó a través de heurísticas con comportamientos y tiempos particulares. El objetivo del trabajo fue determinar si las heurísticas benefician a las metaheurísticas. Para lograr dicho objetivo, se realizó un diseño de experimento donde se comparan todas las combinaciones a través de las métricas de \emph{accuracy} y reducción.

Del experimento realizado se llega a la conclusión de que el uso de heurísticas beneficia principalmente a GGA y SSGA, que al ser combinadas con CNN y RSS, mejoran sus niveles de \emph{accuracy} + reducción. Mientras que MA y CHC no se benefician del uso de heurísticas. Estos detalles permiten inferir que las metaherísticas que poseen herramientas especializadas para la intensificación y diversificación en la búsqueda de buenas soluciones, pueden hallarlas sin importar la población inicial con las que tienen que trabajar; mientras que, aquellas metaheurísticas que carecen de recursos, dependen más de la calidad de la población inicial para llegar a buenas soluciones.

Finalmente, cuando se comparan todas las metaheurísticas, se tiene que las variantes de CHC y MA presentan los mejores valores de \emph{accuracy} y reducción para los conjuntos de los tres tamaños, además, estas metaheurísticas son estadísticamente similares. Por lo tanto, la elección de cuál es la mejor metaheurística para resolver el problema de selección de prototipos se rige por el tiempo de cómputo que toman en terminar. Ya que CHC con población inicial aleatoria es la metaheurística que tarda menos en devolver un resultado, se considera como la mejor. 

Las recomendaciones que se desprenden de esta investigación incluyen: 

\begin{itemize}

\item Probar otras heurísticas y metaheurísticas a las utilizadas en este trabajo.

\item Utilizar otras funciones de distancia distintas a la euclídea con el fin de incorporar conjuntos de datos categóricos.

\item Plantear el problema de selección de prototipos como un problema de optimización multiobjetivo en relación al \emph{accuracy, kappa}, reducción y tiempo para estudiar el comportamiento de los distintos algoritmos frente a la curva pareto obtenida.

\end{itemize}

