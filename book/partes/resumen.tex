\addtotoc{Resumen}
\abstract{
\addtocontents{toc}{\vspace{1em}}

El continuo crecimiento en la cantidad de información procesada y almacenada en bases de datos, producto de los avances en la academia y en la industria, ha traido problemas de escalamiento en la aplicación de algoritmos de minería de datos. Es por eso que se necesita adoptar un método de reducción de datos que sea eficaz y eficiente, es aquí donde tiene relevancia el problema de Selección de Prototipos (PS por sus siglas en inglés), el cual consiste en seleccionar el subconjunto de instancias de menor cardinalidad apartir de un conjunto base, que mantenga o mejore la precisión de clasificación al ser usado como conjunto de entrenamiento \cite{garcia2016data}. Como las metaheurísticas (algoritmos de proposito general enfocados a conseguir ``buenas'' soluciones a un problema determinado) son las candidatas naturales para resolver PS, en este trabajo se busca evaluar si al darles información procesada y filtrada previamente por otros métodos, se puede mejorar el desempeño de las mismas. Para lograr este cometido, se combinan tres metodos de filtrado prevido (CNN, ENN y RSS) con cuatro metaheurísticas (GGA, SSGA, MA y CHC) y se evalúan con respecto a la precisión en la clasificación y la tasa de reducción de instancias. La evaluación experimental indica que sólo SSGA fue beneficiada por el uso de un filtro previo, los resultados individuales de CNN, ENN y RSS no están correlacionados con los resultados de sus combinaciones con las metaheurísticas y además, la versión original de CHC obtiene los mejores resultados bajo las métricas utilizadas.

}

% Las palabras clave son generalmente los nombres de áreas de investigación a
% los cuales está asociado el trabajo. Generalmente son tres o cuatro.
\noindent \begin{small} \textbf{Palabras clave}: reducción de datos, selección de prototipos, heurísticas, metaheurísticas.
\end{small}
