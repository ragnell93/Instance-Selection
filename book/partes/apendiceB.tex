\chapter{Pruebas estadísticas}
\label{Apéndices}
\lhead{Apéndice B. \emph{Apéndice B}}

\begin{table}[]
\centering
\begin{tabular}{l c c c c c c}
\hline
\multirow{3}{*}{\textsc{Algoritmo}}
	& \multicolumn{6}{c}{\textsc{GGA}} \\
	& \multicolumn{2}{c}{\textsc{Accuracy}}
	& \multicolumn{2}{c}{\textsc{Kappa}}
	& \multicolumn{2}{c}{\textsc{Reducción}} \\
 & $W$ & $p$-valor & $W$ & $p$-valor & $W$ & $p$-valor \\
\hline
\hline

GGA & 46 & $3.574 \times 10^{-1}$ &  1 & $1.221 \times 10^{-4}$ & $1.221 \times 10^{-4}$ \\
SGA & 27 & $5.945 \times 10^{-2}$ &  0 & $6.104 \times 10^{-5}$ & $6.104 \times 10^{-5}$ \\
CHC &  6 & $8.545 \times 10^{-4}$ & 14 & $6.714 \times 10^{-3}$ & $6.714 \times 10^{-3}$ \\
PSO &  6 & $8.545 \times 10^{-4}$ &  0 & $6.104 \times 10^{-5}$ & $6.104 \times 10^{-5}$ \\

\hline
\end{tabular}
\caption[Pruebas de \emph{Wilcoxon} entre GGA y las demás metaheurísticas]{Estadístico $W$ y $p$-valor de pruebas de rangos con signo de \emph{Wilcoxon} para determinar si PBIL es mejor que el resto de las metaheurísticas, en función del error de validación y el tamaño de las soluciones encontradas.}
\label{wilcox-res-pbil}
\end{table}